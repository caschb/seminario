\documentclass{article}
\usepackage[T1]{fontenc}
\usepackage[spanish]{babel}
\usepackage[colorlinks]{hyperref}
\usepackage[a4paper, total={6.5in, 10in}]{geometry}
\usepackage{csquotes}
\usepackage[backend=biber]{biblatex}
\addbibresource{refs.bib}

\begin{document}
\title{Avance 1}
\author{Christian Asch}
\date{}
\maketitle
% Centrar el uso de la herramienta en el título y el objetivo principal?
\section{Tema de investigación}
 Para este proyecto se propone la aplicación de bibliotecas de análisis y visualización in-situ en simulaciones sismológicas numéricas.
\section{Título propuesto}
Implementación de una extensión para una simulación sísmica numérica que permita el análisis y visualización in-situ.
\section{Problema a resolver}
Las simulaciones sísmicas son de gran importancia, ya que complementan los datos históricos, mejorando así el análisis para la toma de decisiones en diseño e infraestructura. La literatura clasifica estos trabajos en seis campos distintos \cite{poursartip_large-scale_2020}. De interés para nuestro estudio son aquellos que utilizan métodos numéricos y computacionales para el modelado de sísmos, estas se han beneficiado del avance en la computación de alto rendimiento (CAR) y los sistemas de supercomputación. Un ejemplo es AWP-ODC, una simulación escalable que utiliza el método de diferencias finitas. Este software ha sido utilizado para simular sismos de hasta magnitud 8 con una frecuencia de 2 Hz en el sur de California, abarcando un área de 800 km por 400 km en computadoras de petaescala \cite{Cui2010}, capaces de realizar $10^{15}$ operaciones de punto flotante de 64 bits por segundo.\\
La llegada de las supercomputadoras de exaescala, capaces de realizar $10^{18}$ operaciones de punto flotante de 64 bits por segundo, presenta nuevos desafíos. Si bien se ha incrementado el poder computacional para la investigación científica, algunos componentes no han evolucionado al mismo ritmo. Los cuellos de botella en los sistemas de entrada y salida (E/S) plantean un desafío para el aprovechamiento de recursos, especialmente en análisis posteriores como la creación de visualizaciones. Para abordar este problema se han propuesto soluciones, como el análisis y la visualización in-situ \cite{akira_kageyama_approach_2014}.\\
Un ejemplo es la extensión de la simulación AWP-ODC con la biblioteca de análisis y visualización in-situ, Catalyst. Esta extensión demostró un rendimiento aceptable en comparación con la versión original, con la ventaja adicional de una visualización in-situ, lo que redujo significativamente el tamaño de almacenamiento de la salida del programa \cite{mu_-situ_2019}. Este estudio resalta los resultados prometedores y beneficiosos de utilizar bibliotecas en simulaciones numéricas de sismología para la investigación.\\
Otro aspecto beneficioso del análisis y visualización in-situ es la capacidad de dirigir computacionalmente las simulaciones \cite{Grosset2020}. Tradicionalmente, las visualizaciones y métricas de simulación solo estaban disponibles después de que esta haya finalizado o alcanzado un punto de control. Sin embargo, el análisis y visualización in-situ permite que los resultados estén disponibles constantemente durante la simulación, lo que brinda a los investigadores la oportunidad de tomar decisiones anticipadas, como detener la simulación si comienza a divergir o si cambian las condiciones. Este tipo de interacción con las simulaciones se ha utilizado en otros dominios \cite{Yi2014} y sería valioso para las simulaciones sísmicas.\\
Para este estudio se busca crear una extensión de una simulación existente que sea utilizable por investigadores en sismología, que tenga un rendimiento razonable y les dé la oportunidad de hacer dirección computacional de las simulaciones, validada con escenarios nacionales e internacionales.

\section{Objetivos}
\subsection{Objetivo general}
Construir y evaluar una extensión usable, eficiente e introespectiva para una simulación sísmica numérica que permita el análisis y visualización in-situ de la misma. % No usar "introespectiva" aquí y delimitarlo en la metodología. La palabra extensión es ambigüa en este contexto. Diferenciar del tercer objetivo específico
\subsection{Objetivos específicos}
\begin{enumerate}
  \item Investigar simulaciones en otros dominios que hayan incorporado el análisis in-situ. % la idea de esto es aportar algo extra, no es necesaria para la investigación en sí. Tal vez utilizar la palabra "Caracterizar"
  \item Diseñar una extensión usable, eficiente e introespectiva de análisis y visualización in-situ.
  \item Implementar una extensión usable, eficiente e introespectiva para la simulación utilizando la biblioteca escogida. % Diseñar e implementar pueden ir en un mismo objetivo. 
  \item Evaluar, optimizar y validar la extención implementada. % Especificar lo que significa por "Evaluar", no utilizar la palabra "optimizar"
    % Decir para xxxx ?
\end{enumerate}

\section{Preguntas de investigación}
¿Cómo desarrollar una extensión usable, eficiente e introespectiva para una simulación sísmica numérica que utilice el análisis y visualización in-situ?

\printbibliography

\end{document}
