En esta sección se presentan los trabajos más relevantes para la investigación en curso. Para realizar esta sección se buscaron artículos que trataran el tema del análisis y simulación in-situ aplicado directamente a simulaciones sísmicas. Para analizar estos artículos se tomaron en cuenta los siguientes aspectos:
\begin{itemize}
    \item ¿Se adaptó o modificó una simulación ya existente o se creó una específicamente diseñada con la visualización y el análisis in-situ en mente? (P1)
    \item ¿Se hace uso de bibliotecas externas de in-situ o las técnicas utilizadas están integradas directamente en la simulación? (P2)
    \item ¿El análisis o visualización se realiza en los mismos recursos computacionales en los que se lleva a cabo la simulación o los datos generados son transferidos a algún hardware específico para realizar el procesamiento? (P3)
\end{itemize}

Se encontraron tres trabajos sobre simulaciones sísmicas que hacen uso del análisis y visualización in-situ a distintos grados, en la tabla \ref{tab:related_work} se puede encontrar un resumen de estos.

El trabajo de Yu et al \cite{Yu2006} en el 2006 describe el desarrollo de un framework de visualización in-situ para simular sismos, que adicionalmente permite realizar direccionamiento de la simulación mientras se ejecuta. Este se compone de dos sistemas; el primero es la biblioteca Hercules que realiza, de forma distribuida en una supercomputadora, el mallado del espacio del problema, el particionamiento del problema entre los nodos, la resolución de las ecuaciones diferenciales parciales y finalmente la visualización. El otro componente es el programa QuakeShow, este se utiliza desde una laptop o una computadora de escritorio y sirve para realizar una composición de las imágenes renderizadas en el clúster, mostrarla al usuario y leer las instrucciones de este por medio de gestos con el mouse, por ejemplo hacer zoom o cambiar el ángulo de la cámara.
Los autores probaron el framework simulando un terremoto en Los Ángeles y estiman que si no se hubiese utilizado las técnicas de visualización in-situ hubieran tenido que guardar 250 GB de archivos.

En el artículo de Mu et al \cite{mu_-situ_2019} del 2019, crean la utilidad awp-odc-insitu, basada en la simulación awp-odc-os y el paquete de visualización ParaView con su componente de visualización in-situ, Catalyst. El código original utiliza un diseño en el que los datos son escritos a un buffer que minimiza la cantidad de veces en que los datos son escritos al disco. Para la versión de los autores, se decidió cambiar este diseño ya que lo que se busca generar son visualizaciones en tiempo real. Lo que se hace en esta versión es enviar los datos generados en cada iteración a un adaptador que mapea las estructuras nativas de la simulación a estructuras de VTK utilizadas por esta versión de Catalyst.

Finalmente, en el artículo de Godoy et al \cite{Godoy2020} se habla de la nueva versión de la biblioteca ADIOS y entre sus aplicaciones se menciona que la simulación SPECFEM3D\_GLOBE \cite{Peter_Forward_and_adjoint_2011} hace uso de esta, sin embargo no existe un artículo correspondiente donde se detalle esta implementación. Revisando la documentación del programa se nota que ADIOS se ofrece como una alternativa para realizar operaciones de E/S con archivos.

\begin{table}[tbp]
    \begin{tabular}{|l|l|l|l|}
        \hline
                         & P1                            & P2        & P3     \\ \hline
        Hercules         & Simulación hecha para in-situ & Integrada & Mismos \\ \hline
        awp-os-insitu    & Simulación modificada         & Externa   & Mismos \\ \hline
        SPECFEM3D\_GLOBE & Simulación modificada         & Externa   & Mismos \\ \hline
    \end{tabular}
    \label{tab:related_work}
    \caption{Resumen de los trabajos encontrados}
\end{table}